%% ----------------------------------------------------------------
%% Thesis.tex -- MAIN FILE (the one that you compile with LaTeX)
%% ---------------------------------------------------------------- 

% Set up the document
\documentclass[a4paper, 11pt, twoside, openright]{uis-bachelor-master}  % Use the "Thesis" style, based on the ECS Thesis style by Steve Gunn
\graphicspath{{Figures/}}  % Location of the graphics files (set up for graphics to be in PDF format)

% Include any extra LaTeX packages required
\usepackage[square, numbers, comma, sort&compress]{natbib}  % Use the "Natbib" style for the references in the Bibliography
\usepackage{verbatim}  % Needed for the "comment" environment to make LaTeX comments
\usepackage{vector}  % Allows "\bvec{}" and "\buvec{}" for "blackboard" style bold vectors in maths
\usepackage{algorithm}
\usepackage{algorithmic}
\usepackage[T1]{fontenc}
\usepackage{subfigure}
\usepackage{venturis2}
\usepackage{lmodern}
\usepackage{textcomp}    % solve issues with lmodern
\usepackage{amsfonts}
\usepackage{amsmath}
\usepackage{amsthm}
\usepackage{microtype}   % better typesetting with pdfLaTeX
\usepackage{ps4pdf}
\PSforPDF{
  \usepackage{pstricks}
}
\usepackage[compact]{titlesec}
\usepackage{booktabs}
\usepackage{sectsty}     % section titles in specified font face


\allsectionsfont{\sffamily}
\numberwithin{algorithm}{chapter}
\setcounter{secnumdepth}{2}
\setcounter{tocdepth}{2}
\renewcommand{\captionlabelfont}{\sffamily\bfseries}
\newtheorem{thm}{Theorem}
\renewcommand{\algorithmicrequire}{\textbf{Input:}}
\renewcommand{\algorithmicensure}{\textbf{Output:}}
\hypersetup{urlcolor=blue, colorlinks=true}  % Colours hyperlinks in blue, but this can be distracting if there are many links.

\listfiles

%\usepackage[draft]{hyperref}
%\usepackage[hyperfootnotes=false,plainpages=false]{hyperref}
%% ----------------------------------------------------------------
\begin{document}
\frontmatter	  % Begin Roman style (i, ii, iii, iv...) page numbering

% Set up the Title Page
\title  {Generative adversarial networks for improving fake-news detection}
\authors  {\texorpdfstring
            {Khoa Nguyen Le}
            {Khoa Nguyen Le}
            }
\addresses  {\groupname\\\deptname\\\univname}  % Do not change this here, instead these must be set in the "Thesis.cls" file, please look through it instead
\date       {\today}
\subject    {}
\keywords   {}

\maketitle
%% ----------------------------------------------------------------

\setstretch{1.3}  % It is better to have smaller font and larger line spacing than the other way round

% Define the page headers using the FancyHdr package and set up for one-sided printing
%\fancyhead{}  % Clears all page headers and footers
%\rhead{\thepage}  % Sets the right side header to show the page number
%\lhead{}  % Clears the left side page header

% \pagestyle{fancy}  % Finally, use the "fancy" page style to implement the FancyHdr headers
% \fancyhead[RE,LO]{\sffamily\bfseries\nouppercase{\rightmark}}
% \fancyhead[LE,RO]{\thepage}
% %% ----------------------------------------------------------------
% % Declaration Page required for the Thesis, your institution may give you a different text to place here
% \Declaration{

% \addtocontents{toc}{\vspace{1em}}  % Add a gap in the Contents, for aesthetics

% I, Vinay Setty, declare that this thesis titled, `Efficiently Identifying Interesting Time Points in Text Archives' and the work presented in it are my own. I confirm that:

% \begin{itemize} 
% \item[\tiny{$\blacksquare$}] This work was done wholly or mainly while in candidature for a master's degree at this University.
  
% \item[\tiny{$\blacksquare$}] Where I have consulted the published work of others, this is always clearly attributed.
 
% \item[\tiny{$\blacksquare$}] Where I have quoted from the work of others, the source is always given. With the exception of such quotations, this thesis is entirely my own work.
 
% \item[\tiny{$\blacksquare$}] I have acknowledged all main sources of help.

% \end{itemize}
%  \vspace{5pt}
% Signed:\\
% \rule[1em]{25em}{0.5pt}  % This prints a line for the signature
 
% Date:\\
% \rule[1em]{25em}{0.5pt}  % This prints a line to write the date
% }
% \clearpage  % Declaration ended, now start a new page
\pagestyle{empty}
\mbox{}
\clearpage
%% ----------------------------------------------------------------
% The "Funny Quote Page"
\pagestyle{empty}  % No headers or footers for the following pages

\null\vfill
% Now comes the "Funny Quote", written in italics
\textit{``Our intelligence is what makes us human, and AI is an extension of that quality.''}

\begin{flushright}
Yann LeCun\end{flushright}

\vfill\vfill\vfill\vfill\vfill\vfill\null
\clearpage  % Funny Quote page ended, start a new page
%% ----------------------------------------------------------------
\pagestyle{empty}
\mbox{}
\clearpage
% The Abstract Page
\addtotoc{Abstract}  % Add the "Abstract" page entry to the Contents
\abstract{
\addtocontents{toc}{\vspace{1em}}  % Add a gap in the Contents, for aesthetics
The disinformation news in media channels such as social media websites or online newspaper have becoming a big challenge
for many organizations, goverments and scientific researchers. In connection to fake news, the political bias (left-sided
or right-sided) of the news articles are recently receiving more attention. In this thesis, we leverage the Adversarially 
Regularized AutoEncoder (ARAE) model, which enhances the adversarial autoencoder (AAE) by learning a parameterized prior
as a Generative Adversarial Networks (GAN) to classify fake news and generate the bias-flipped headlines. We perform the 
experiments with multiple datasets then discuss how these approaches contribute to improving fake-news detection.
}

\clearpage  % Abstract ended, start a new page
%% ----------------------------------------------------------------
\pagestyle{empty}
\mbox{}
\clearpage
\setstretch{1.3}  % Reset the line-spacing to 1.3 for body text (if it has changed)

% The Acknowledgements page, for thanking everyone
\acknowledgements{
\addtocontents{toc}{\vspace{1em}}  % Add a gap in the Contents, for aesthetics
First, I would like to thank advisor Vinay Jayarama Setty for giving me lots of advises, datasets and also computing resources
throughout the thesis process. I would also like to thank Nina Egeland for her administrative support. 
Second, I thank the technical team at University of Stavanger for allocating the GPU servers to train our models.

Khoa Nguyen Le, Stavanger, June 2020.

}
\clearpage  % End of the Acknowledgements
%% ----------------------------------------------------------------

\pagestyle{fancy}  %The page style headers have been "empty" all this time, now use the "fancy" headers as defined before to bring them back


%% ----------------------------------------------------------------
%\lhead{\emph{Contents}}  % Set the left side apge header to "Contents"
\tableofcontents  % Write out the Table of Contents

%% ----------------------------------------------------------------
\setstretch{1.5}  % Set the line spacing to 1.5, this makes the following tables easier to read
\clearpage  % Start a new page
\lhead{\emph{Abbreviations}}  % Set the left side page header to "Abbreviations"

\listofsymbols{ll}  % Include a list of Abbreviations (a table of two columns)
{
\textbf{Acronym} & \textbf{W}hat (it) \textbf{S}tands \textbf{F}or \\
\textbf{LAH} & \textbf{L}ist \textbf{A}bbreviations \textbf{H}ere \\
\addtocontents{toc}{\vspace{1em}}  % Add a gap in the Contents, for 
}


% aesthetics


% }
% ----------------------------------------------------------------
\clearpage  %Start a new page
\lhead{\emph{Symbols}}  % Set the left side page header to "Symbols"
\listofnomenclature{lll}  % Include a list of Symbols (a three column table)
{
symbol & name & unit \\
$a$ & distance & m \\
$P$ & power & W (Js$^{-1}$) \\
& & \\ % Gap to separate the Roman symbols from the Greek
$\omega$ & angular frequency & rads$^{-1}$ \\
}
%% ----------------------------------------------------------------
% End of the pre-able, contents and lists of things
% Begin the Dedication page

\setstretch{1.3}  % Return the line spacing back to 1.3

%\pagestyle{empty}  % Page style needs to be empty for this page

\addtocontents{toc}{\vspace{2em}}  % Add a gap in the Contents, for aesthetics


%% ----------------------------------------------------------------
\mainmatter	  % Begin normal, numeric (1,2,3...) page numbering
\pagestyle{fancy}  % Return the page headers back to the "fancy" style

% Include the chapters of the thesis, as separate files
% Just uncomment the lines as you write the chapters

% Chapter 1

\chapter{Introduction} % Write in your own Chapter title
\label{Chapter1}
%\lhead{Chapter 1. \emph{Introduction}} % Write in your own Chapter title to set the page header


\section{Motivation}

\section{Problem Definition}

\section{Usecases/Examples}

\section{Challenges}

\section{Contributions}

\section{Outline}
 % Introduction

% Chapter 2

\chapter{Related Work} % Write in your own Chapter title
\label{Chapter2}
 % Related Work

% Chapter 3

\chapter{Solution Approach} % Write in your own Chapter title
\label{Chapter3}
%\lhead{Chapter 3. \emph{Rank-Aware Index Structures for Ranked Intervals}} % Write in your own Chapter title to set the page header


\section{Introduction}
 

\section{Existing Approaches/Baselines}
\section{Analysis}
\section{Proposed Solution} % Survey

% Chapter 1

\chapter{Experimental Evaluation} % Write in your own chapter title
\label{Chapter4}
%\lhead{Chapter 1. \emph{Experimental Evaluation}} % Write in your own chapter title to set the page header


\section{Experimental Setup and Data Set}
\section{Experimental Results}
 % Main Approach

% Chapter 5

\chapter{Discussion} % Write in your own chapter title
\label{Chapter5}






 % Experiments

% Chapter 1

\chapter{Conclusion and Future Directions} % Write in your own chapter title
\label{Chapter6}
%\lhead{Chapter 1. \emph{Conclusion And Future Work}} % Write in your own chapter title to set the page header





 % Conclusion and Future Work

%\input{./Chapters/Chapter7} % Conclusion

%% ----------------------------------------------------------------
% Now begin the Appendices, including them as separate files

\addtocontents{toc}{\vspace{2em}} % Add a gap in the Contents, for aesthetics

%% ----------------------------------------------------------------
%\lhead{\emph{List of Figures}}  % Set the left side page header to "List if Figures"
\listoffigures  % Write out the List of Figures

%% ----------------------------------------------------------------
%\lhead{\emph{List of Tables}}  % Set the left side page header to "List of Tables"
\listoftables  % Write out the List of Tables

\appendix % Cue to tell LaTeX that the following 'chapters' are Appendices

%% Appendix A

\chapter{Appendix Title Here}
\label{AppendixA}
%\lhead{Appendix A. \emph{Appendix Title Here}}

Write your Appendix content here.	% Appendix Title

%\input{./Appendices/AppendixB} % Appendix Title

%\input{./Appendices/AppendixC} % Appendix Title

\addtocontents{toc}{\vspace{2em}}  % Add a gap in the Contents, for aesthetics
\backmatter

%% ----------------------------------------------------------------
\label{Bibliography}
\lhead{\emph{Bibliography}}  % Change the left side page header to "Bibliography"
\bibliographystyle{unsrtnat}  % Use the "unsrtnat" BibTeX style for formatting the Bibliography
\bibliography{Bibliography}  % The references (bibliography) information are stored in the file named "Bibliography.bib"

\end{document}  % The End
%% ----------------------------------------------------------------
