% Chapter 1

\chapter{Introduction} % Write in your own Chapter title
\label{Chapter1}
%\lhead{Chapter 1. \emph{Introduction}} % Write in your own Chapter title to set the page header


\section{Motivation}
To date, fake news can be found on any media outlets, especially social media like Facebook, Twitter and online newspapers. Thus, there's an increasing
number of public organizations and scientific researchers trying to tackle this problem~\cite{perez2017automatic}. 

While in some previous researches, the most common computational way is automatic fact-checking and relied on the news sources~\cite{perez2017automatic}, 
our approach puts more focus on the writing style of the fake news, in particular, political-biased claims. The work from~\cite{potthast2018stylometric} 
have shown that the articles' writing styles could be use to detect the hyperpartisan news from the more balanced ones. The style of left-sided and right-sided 
articles appear to be more related than they have with mainstream's style or to the opposite side's style. Considering an example from~\cite{chen-etal-2018-learning}:

\textit{Why Trump is right in recognizing Jerusalem as
Israel’s capital}

\textit{Trump is making a huge mistake on Jerusalem}

The two headlines above are from Fox News and New York Times respectively and are about the same event of Donald Trump recognizing Jerusalem as the
Israel's capital. The different stance in those headlines can lead the readers to very different impressions.

In order to learn the style of news headlines or claims, we unaligned textual style transfer using the adversarially regularized autoencoder (ARAE). This model 
is based on Wasserstein autoencoder (WAE) framework~\cite{Tolstikhin2018WassersteinA} and generative adversarial networks (GAN)~\cite{NIPS2014_5423} which has 
contributed a major advancement in text generation tasks. We also use GloVe~\cite{glove} pre-trained vectors as word embedding to replace for the model's word embedding
which only relied on the word indices in the vocabulary.

The experiments have been performed on different datasets:
\begin{itemize}
  \item The Webis Bias Flipper 2018~\cite{stein:2018y} comprises 2781 events from allsides.com as of June 1st, 2012 till February 10, 2018. We use this dataset to generate
the right-sided and left-sided bias articles titles.
  \item The Hyperpartisan News dataset~\cite{johannes_kiesel_2018_1489920}
\end{itemize}
\section{Problem Definition}

\section{Usecases/Examples}

\section{Challenges}

\section{Contributions}

\section{Outline}
